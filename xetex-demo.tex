\documentclass[11pt]{article}
\ProvidesPackage{zhfontcfg} 
%\usepackage[top=2.5cm,bottom=2.5cm,left=2.5cm,right=2.5cm]{geometry} %页面设置
%\usepackage[colorlinks,linkcolor=red,hyperindex,CJKbookmarks,dvipdfm]{hyperref}  %超级连接
\usepackage[cm-default]{fontspec}               %[cm-default]选项主要用来解决使用数学环境时数学符号不能正常显示的问题 
\usepackage{xunicode,xltxtra}  
\usepackage{graphicx}
\usepackage{amsmath}
\defaultfontfeatures{Mapping=tex-text}          %如果没有它,会有一些 tex 特殊字符无法正常使用,比如连字符。 
%\setmainfont[BoldFont=黑体]{宋体}               % 使用系统默认字体
\XeTeXlinebreaklocale "zh"                      % 针对中文进行断行
\XeTeXlinebreakskip = 0pt plus 1pt minus 0.1pt  % 给予TeX断行一定自由度
\linespread{1.3}                                % 1.5倍行距
%将系统字体名映射为逻辑字体名称,主要是为了维护的方便  
\newcommand\fontnamehei{Microsoft YaHei}  
\newcommand\fontnamesong{SimSun}  
\newcommand\fontnamekai{AR PL KaitiM GB}  
\newcommand\fontnamemono{DejaVu Sans Mono}  
\newcommand\fontnameroman{Times New Roman}  
%%设置常用中文字号,方便调用  
\newcommand{\erhao}{\fontsize{22pt}{\baselineskip}\selectfont}  
\newcommand{\xiaoerhao}{\fontsize{18pt}{\baselineskip}\selectfont}  
\newcommand{\sanhao}{\fontsize{16pt}{\baselineskip}\selectfont}  
\newcommand{\xiaosanhao}{\fontsize{15pt}{\baselineskip}\selectfont}  
\newcommand{\sihao}{\fontsize{14pt}{\baselineskip}\selectfont}  
\newcommand{\xiaosihao}{\fontsize{12pt}{\baselineskip}\selectfont}  
\newcommand{\wuhao}{\fontsize{10.5pt}{\baselineskip}\selectfont}  
\newcommand{\xiaowuhao}{\fontsize{9pt}{\baselineskip}\selectfont}  
\newcommand{\liuhao}{\fontsize{7.5pt}{\baselineskip}\selectfont}  
%设置文档正文字体为宋体  
\setmainfont[BoldFont=\fontnamehei]{\fontnamesong}  
\setsansfont[BoldFont=\fontnamehei]{\fontnamekai}  
\setmonofont{\fontnamemono}  
%楷体  
\newfontinstance\KAI {\fontnamekai}  
\newcommand{\kai}[1]{{\KAI#1}}  
%黑体  
\newfontinstance\HEI{\fontnamehei}  
\newcommand{\hei}[1]{{\HEI#1}}
%英文  
\newfontinstance\ENF{\fontnameroman}  
\newcommand{\en}[1]{\,{\ENF#1}\,}  
%图表更名设置
\renewcommand{\figurename}{图}
\renewcommand{\tablename}{表}

\begin{document}
\title{我有一个漂亮得不得了的表姐}
\author{周羽}
\maketitle
\tableofcontents



% 一直听说 Sublime Text 2 是神器,电脑里也一直装着,但对它的印象一直停留在“好用的代码编辑器”的层面。因为自己也不是码农,没有太深入的了解。直到最近尝试学习 LaTeX ,发现 Sublime Text 能做的事真不少,而且各种 Package 插件的加入更使得 Sublime Text 成为了真正的神器。
% 说到 LaTeX,开始用的是 WinEdt,不得不说,它功能非常强大,各种能想到的想不到的功能都有,各种按钮看的是眼花缭乱,这也使它看起来臃肿而不优雅。Sublime Text 一直是优雅编程的代名词,在 Sublime Text 里写 LaTeX 也是一种享受。当然,安装 LaTeXTools Package 是实现这一切的基础,如何安装 Package 的问题这里不讲,具体可 Google 和参考这里。同时,假定你也安装了 CTeX,当然,其他的 TeX 系统也可参考本文。
% 这样,我们 LaTeX 写作环境就已经搭建完毕,可以通过如下方法进行“优雅”的写作:
% 新建文档,Ctrl + Shift + P调出命令提示,输入 Set Syntax: LaTeX,回车;
% 按照 LaTeX 方法写作;
% Ctrl + B 即可编译(Build) PDF。
% 但是,LaTeXTools 默认使用 PDFTeX 进行编译,它对于中文的支持好像并不完备,一些学校的模板也会出现这样那样的问题,所以,推荐使用 XeTeX/XeLaTeX 进行编译。到这里,我们的需求就变为:如何将 Windows1 下的 LaTeXTools 默认编译工具改为 XeTeX/XeLaTeX?
% 找到 ...\Sublime Text 2\Data\Packages\LaTeXTools\LaTeX.sublime-build 文件,用文本编辑器打开;
% 将
% "cmd": ["texify","-b","-p","--tex-option=\"--synctex=1\""]
% 修改为
% "cmd": ["texify","-b","-p","--engine=xetex","--tex-option=\"--synctex=1\""]
% 保存重新启动 Sublime Text 编译就转换为 XeTeX 引擎。



% http://bbs.ctex.org/forum.php?mod=viewthread&tid=43236&extra=&page=1
% 最近写毕业论文,要符合学校的变态格式(很花哨的变换字体样式和大小)。以前一直用CTeX,想到那么麻烦的字体设置就放弃了。用OpenOffice折腾了一天弄了个模板,生成的PDF觉得惨不忍睹。想到加上内容后还要微调的痛苦,还是想回到用TeX。读了lyanry写的那篇“XeTeX中文排版之胡言乱语”   xetex-tutorial.pdf (222.91 KB, 下载次数: 4358) ,觉得XeTeX非常好。尝试了一下,很容易就实现了以前在CTeX里面十二万分痛苦的字体配置,而且生成的PDF体积小、效果好(估计是因为使用了Adobe的高质量OTF字体)。现在已经完全删除了CTeX套装(包括type1字体包),全面转投XeTeX怀抱。

% 各位如果还在和CTeX肉搏的话,建议试试XeTeX。windows用户推荐用instanton做的MiCTeX 2.7版。

% 感谢lyanry,你的那篇教程让我很快上手XeTeX;
% 感谢孙老师,你贡献的xeCJK宏包我虽然没有直接用,但是里面的说明文件让我学到很多;
% 感谢instanton,你的小巧但完全的MiCTeX让我省去不少麻烦。

% 把结果输出成一个UTF-8编码的文件,用编辑器打开,如果这个编辑器带有行排序的功能,你可以把字体列表排序一下,方便查找。至于简单的源文件:

% \documentclass[12pt]{article}
% \usepackage[cm-default]{fontspec}
% \defaultfontfeatures{Mapping=tex-text}
% \usepackage{xunicode}% provides unicode character macros
% \usepackage{xltxtra} % provides some fixes/extras
% \begin{document}
% \setmainfont{SimSun}
% 这是一个华丽的测试
% \end{document}


% 试试下面这个。导言区比较简单。另外,请检查你的tex文件是否用utf8编码保存了?
% \documentclass[12pt,a4paper]{article} 

% \usepackage{fontspec}
% \setmainfont[BoldFont=黑体]{宋体}               % 使用系统默认字体

% \XeTeXlinebreaklocale "zh"                      % 针对中文进行断行
% \XeTeXlinebreakskip = 0pt plus 1pt minus 0.1pt  % 给予TeX断行一定自由度
% \linespread{1.5}                                % 1.5倍行距

% \begin{document}

% \section{第八章 没落的贵族—摩托罗拉}

% 美国过去未曾有过贵族,今后也不会有。无论是巨富盖茨或者是年轻美貌、聪明而富有的女继承人伊万卡·特朗普都不是任何意义上的贵族。实际上贵族这个词在整个西方本身就是一个没落的词汇,虽然在东方一些人或许沉迷在贵族梦中。但是,贵族在历史上曾经实实在在地出现过。如果我们认为公司之中也有所谓的贵族,摩托罗拉无疑可以算是一个。

% 曾几何时,摩托罗拉就是无线通信的代名词,同时它还是技术和品质的结晶。甚至就在二十年前,摩托罗拉还在嘲笑日本品质的代表索尼,认为后者的质量只配做体育用品。今天,虽然摩托罗拉的产品从品质上讲仍然傲视同类产品,但是就像一个戴着假发拿着手杖的贵族,怎么也无法融入时尚的潮流。

% \end{document}

% xelatex里面有一个fontspec.pdf讲的很清楚啊。



















\end{document}